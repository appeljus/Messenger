\documentclass{article}

\usepackage[dutch]{babel}

\title{Integration Project}
\author{Kimberly Hengst \emph{1305093}\\ Tim Sonderen \emph{1465252}\\ Kevin Hetterscheid \emph{1490443}\\ Martijn de Bijl \emph{1470108}}
\date{\today}

\begin{document}

\maketitle

\newpage

\tableofcontents

\section{Inleiding}
Dit is ons verslag van het project van de derde module. Voor het project hebben wij een chat applicatie gemaakt. In deze applicatie kan een gebruiker in een ad-hoc netwerk met andere gebruikers berichten uitwisselen. De applicatie ondersteunt maximaal vier gebruikers. Een gebruiker kan ervoor kiezen om met één persoon, of drie personen te chatten. De berichten zijn beveiligd met behulp van encryptie. Hierdoor kunnen alleen de andere gebruikers de berichten uitlezen. Als twee gebruikers alleen willen chatten, is dit voor de andere gebruikers geëncrypt. Het netwerk waarop de applicatie draait is een ad-hoc netwerk. Dit betekent dat het netwerk geen router of server nodig heeft. De computers  maken een eigen netwerk via Wi-Fi.
\\
In het verslag zullen we het proces van het ontwerpen van de applicatie, en de keuzes die we daarbij gemaakt hebben uitleggen. Als eerste bekijken we het systeem ontwerp, hier zal de basis uitgelegd worden. Daarna zal de implementatie van de beveiliging uitgelegd worden. Als laatste zal het test proces beschreven worden.

\newpage

\section{Systeem ontwerp}
De gebruiker kan in een user interface berichten typen en versturen. Deze berichten worden dan in een pakket opgeslagen, en het pakket wordt verstuurd. Het versturen gebeurt via een multicast socket. Dit betekent dat het bericht naar iedereen verstuurd wordt. Als een socket een pakket ontvangt, wordt deze in de GUI weergeven. 

\subsection{Pakketjes}
Voor het versturen van de pakketjes gebruikt multicast socket een Datagram packet. Deze gebruiken wij ook voor het versturen van data. Een datagram packet heeft een byte array met data, een IP adres van de ontvanger, en een poort van de ontvanger. In mulitcast wordt een pakketje dus naar iedereen verstuurd, het IP adres is dan ook het adres van een multicast network, een IP van de IP-klasse D. De poort is een standaard poort waar wij voor hebben gekozen. In de data van de datagram packet worden nog enkele andere velden aangewezen. %Welke velden staan waar voor?
Packets die voor utility worden gebruikt, beginnen met hun naam (BROADCAST, NAME\_IN\_USE, etc) tussen blokhaken. Namen mogen geen [ of ] bevatten, dus kan er zo geen error in komen.


\subsection{Bestanden verzenden}
%TODO

\subsection{Pakket verlies}
%TODO
%Hoe zorgen we ervoor dat als er pakket verlies optreedt, dat het pakket opnieuw verzonden wordt. 
%We gebruiken geen ACK's, maar NACK's. Hierdoor ontstaat er geen congestion als je met veel mensen gaat chatten, door de vele ACK's die gestuurd worden.

\subsection{User interface}
De gebruiker kan via de user interface met het programma communiceren. De interface bestaat uit een aantal elementen:
\begin{itemize}
\item Tekstveld
\item Typveld
\item Een knop voor het verzenden van tekst en bestanden
\item Een knop voor privé chat
\item Een knop voor het afsluiten van het applicatie
\end{itemize}

\section{Beveiling}
Om te zorgen dat niet iedereen de inhoudt van de pakketten kan lezen, gebruiken we encryptie. 

\subsection{Encryptie}
Voor het encryptie gebruiken we symetrische sleutel encryptie. Dit betekent dat de clients een sleutel delen, waarmee ze hun berichten kunnen coderen en decoderen.  

\section{Test plan}


\subsection{Test cases}
%Voor elke case moet beschreven zijn:
%		Wat je van het systeem verwacht
%		Wat de preciese test opstelling is. Aan de hand hiervan moet de test herhaal kunnen worden
%		Hoe het test scenario er uit ziet
%		Welke performance we van het systeem verwachten


\end{document}
