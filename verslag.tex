\documentclass{article}

\usepackage[dutch]{babel}

\title{Integration Project}
\author{Kimberly Hengst \\ Tim Sonderen \\ Kevin Hetterscheid \\ Martijn de Bijl}
\date{\today}

\begin{document}

\maketitle

\newpage

\tableofcontents

\section{Inleiding}
Dit is ons verslag van het project van de derde module. Voor het project hebben wij een chat applicatie gemaakt. In deze applicatie kan een gebruiker in een ad-hoc netwerk met andere gebruikers berichten uitwisselen. De applicatie ondersteunt maximaal vier gebruikers. Een gebruiker kan ervoor kiezen om met één persoon, of drie personen te chatten. De berichten zijn beveiligd met behulp van encryptie. Hierdoor kunnen alleen de andere gebruikers de berichten uitlezen. Als twee gebruikers alleen willen chatten, is dit voor de andere gebruikers geëncrypt. Het netwerk waarop de applicatie draait is een ad-hoc netwerk. Dit betekent dat het netwerk geen router of server nodig heeft. De computers  maken een eigen netwerk via Wi-Fi.
\\
In het verslag zullen we het proces van het ontwerpen van de applicatie, en de keuzes die we daarbij gemaakt hebben uitleggen. Als eerste bekijken we het systeem ontwerp, hier zal de basis uitgelegd worden. Daarna zal de implementatie van de beveiliging uitgelegd worden. Als laatste zal het test proces beschreven worden.

\newpage

\section{Systeem ontwerp}
De gebruiker kan in een user interface berichten typen en versturen. Deze berichten worden dan in een pakket opgeslagen, en het pakket wordt verstuurd. Het versturen gebeurt via een multicast socket. Dit betekent dat het bericht naar iedereen verstuurd worden, en 
\subsection{Pakketjes}

\subsection{Pakket verlies}

\section{Beveiling}
\subsection{Encryptie}


\section{Testing}
\subsection{Test methodes}

\end{document}